\documentclass[a4paper,12pt]{article}
\usepackage[utf8]{inputenc}
\usepackage[italian]{babel}
\usepackage{xcolor}
\usepackage{listings}
\usepackage{graphicx}
\usepackage{hyperref}
\usepackage{geometry}
\usepackage{caption}
\geometry{margin=2.5cm}

\title{Manuale Tecnico\\ \large BookRecommender}
\author{Babini Ariele}
\date{\today}

\lstset{
  language=Java,
  basicstyle=\ttfamily\small,
  keywordstyle=\color{blue},
  commentstyle=\color{gray},
  stringstyle=\color{red},
  breaklines=true,
  captionpos=b
}

\begin{document}

    \maketitle
    \tableofcontents
    \newpage

    \section{Introduzione}
    Questo manuale descrive dettagliatamente la struttura tecnica e progettuale del sistema \textbf{BookRecommender}. Il documento è rivolto a sviluppatori e manutentori del sistema, fornendo una panoramica approfondita delle scelte architetturali, algoritmiche e strutturali adottate.

    \section{Struttura dell'Applicazione}
    L'applicazione è organizzata secondo un'architettura modulare in pacchetti. I principali componenti sono:
    \begin{itemize}
    \item \texttt{controller} – gestione logica dell'applicazione
    \item \texttt{model} – rappresentazione dei dati (libri, utenti, raccomandazioni)
    \item \texttt{view} – interfaccia grafica (JavaFX/Swing)
    \item \texttt{utils} – classi di supporto
    \end{itemize}

    \begin{figure}[h]
    \centering
    \fbox{\rule{0pt}{3cm} \rule{5cm}{0pt}} % placeholder for image
    % \includegraphics[width=0.9\textwidth]{diagramma_classi.png}
    \caption{Diagramma delle classi principali (UML)}
    \end{figure}

    \section{Scelte Progettuali}
    \subsection{Diagrammi UML}
    Il sistema è basato su un modello ad oggetti. Di seguito è riportato il diagramma delle classi e delle interazioni principali.
    \begin{itemize}
    \item Class Diagram
    \item Sequence Diagram (per l’uso della raccomandazione)
    \end{itemize}

    \subsection{Schema ER}
    Nel caso di uso di una base di dati:
    \begin{figure}[h]
    \centering
    % \includegraphics[width=0.8\textwidth]{schema_er.png}
    \caption{Schema ER: utenti, libri, raccomandazioni}
    \end{figure}

    \section{Scelte Architetturali}
    L’applicazione è costruita secondo il pattern MVC:
    \begin{itemize}
    \item Separazione tra logica, presentazione e gestione dati
    \item Facilità di estensione e testabilità
    \end{itemize}
    Altri aspetti architetturali:
    \begin{itemize}
    \item Modularità (uso dei package)
    \item Dipendenze gestite tramite Maven
    \item JavaFX come UI framework
    \end{itemize}

    \section{Strutture Dati Utilizzate}
    Sono state utilizzate strutture dati standard della Java Collection Framework:
    \begin{itemize}
    \item \texttt{List<Book>} per collezioni di libri
    \item \texttt{Map<String, List<Book>>} per raccomandazioni per utente
    \item \texttt{Set<String>} per evitare duplicati
    \end{itemize}

    \section{Scelte Algoritmiche}
    Il sistema di raccomandazione può includere (esempi):
    \begin{itemize}
    \item Similarità basata su contenuto (coseno tra vettori di parole chiave)
    \item Filtraggio collaborativo semplice (similitudine tra utenti)
    \item Ordinamento per punteggio
    \end{itemize}

    \begin{lstlisting}[caption={Algoritmo semplificato di raccomandazione}]
    public List<Book> recommend(String userId) {
        List<Book> liked = getLikedBooks(userId);
        Map<Book, Double> scores = new HashMap<>();
        for (Book book : allBooks) {
            double score = computeSimilarity(book, liked);
            scores.put(book, score);
        }
        return scores.entrySet().stream()
            .sorted(Map.Entry.comparingByValue(Comparator.reverseOrder()))
            .limit(5)
            .map(Map.Entry::getKey)
            .collect(Collectors.toList());
    }
    \end{lstlisting}

    \section{Gestione dei File}
    Il sistema carica/salva file di dati nei seguenti formati:
    \begin{itemize}
    \item \texttt{CSV} per dati tabellari (libri, utenti)
    \item \texttt{JSON} opzionale per configurazioni
    \end{itemize}

    \begin{verbatim}
    src/
    |-- main/
    |    |-- resources/
    |         |-- data/
    |              |-- books.csv
    |              |-- users.csv
    \end{verbatim}

    \section{Pattern Utilizzati}
    \subsection{Singleton}
    Utilizzato per gestire l'accesso centralizzato al sistema di raccomandazione.
    \begin{lstlisting}[caption={Pattern Singleton per RecommendationEngine}]
    public class RecommendationEngine {
        private static RecommendationEngine instance;
        private RecommendationEngine() {}

        public static synchronized RecommendationEngine getInstance() {
            if (instance == null)
                instance = new RecommendationEngine();
            return instance;
        }
    }
    \end{lstlisting}

    \subsection{MVC}
    \begin{itemize}
    \item Model: classi \texttt{Book}, \texttt{User}
    \item View: classi GUI JavaFX
    \item Controller: logica di interazione
    \end{itemize}

    \section{Documentazione JavaDoc}
    Tutte le classi sono documentate con JavaDoc. Esempio:
    \begin{lstlisting}[caption={JavaDoc per classe Book}]
    /**
    * Classe che rappresenta un libro.
    * Contiene titolo, autore, genere e valutazione.
    */
    public class Book {
        private String title;
        private String author;
        private String genre;
        private double rating;
        // costruttori, metodi getter/setter, ecc.
    }
    \end{lstlisting}

    \section{Conclusione}
    Il sistema BookRecommender è progettato con modularità e chiarezza per facilitarne l’estensione e la manutenzione. La documentazione tecnica, insieme alla JavaDoc generata automaticamente, permette un agevole accesso alle componenti del codice.

\end{document}